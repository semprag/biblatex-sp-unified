% Options for packages loaded elsewhere
\PassOptionsToPackage{unicode}{hyperref}
\PassOptionsToPackage{hyphens}{url}
\PassOptionsToPackage{dvipsnames,svgnames,x11names}{xcolor}
%
\documentclass[
]{article}
\usepackage{amsmath,amssymb}
\usepackage{lmodern}
\usepackage{iftex}
\ifPDFTeX
  \usepackage[T1]{fontenc}
  \usepackage[utf8]{inputenc}
  \usepackage{textcomp} % provide euro and other symbols
\else % if luatex or xetex
  \usepackage{unicode-math}
  \defaultfontfeatures{Scale=MatchLowercase}
  \defaultfontfeatures[\rmfamily]{Ligatures=TeX,Scale=1}
\fi
% Use upquote if available, for straight quotes in verbatim environments
\IfFileExists{upquote.sty}{\usepackage{upquote}}{}
\IfFileExists{microtype.sty}{% use microtype if available
  \usepackage[]{microtype}
  \UseMicrotypeSet[protrusion]{basicmath} % disable protrusion for tt fonts
}{}
\makeatletter
\@ifundefined{KOMAClassName}{% if non-KOMA class
  \IfFileExists{parskip.sty}{%
    \usepackage{parskip}
  }{% else
    \setlength{\parindent}{0pt}
    \setlength{\parskip}{6pt plus 2pt minus 1pt}}
}{% if KOMA class
  \KOMAoptions{parskip=half}}
\makeatother
\usepackage{xcolor}
\IfFileExists{xurl.sty}{\usepackage{xurl}}{} % add URL line breaks if available
\IfFileExists{bookmark.sty}{\usepackage{bookmark}}{\usepackage{hyperref}}
\hypersetup{
  pdftitle={biblatex-unified},
  pdfauthor={Kai von Fintel (fintel@mit.edu)},
  colorlinks=true,
  linkcolor={Maroon},
  filecolor={Maroon},
  citecolor={Blue},
  urlcolor={Blue},
  pdfcreator={LaTeX via pandoc}}
\urlstyle{same} % disable monospaced font for URLs
\usepackage{color}
\usepackage{fancyvrb}
\newcommand{\VerbBar}{|}
\newcommand{\VERB}{\Verb[commandchars=\\\{\}]}
\DefineVerbatimEnvironment{Highlighting}{Verbatim}{commandchars=\\\{\}}
% Add ',fontsize=\small' for more characters per line
\usepackage{framed}
\definecolor{shadecolor}{RGB}{255,255,255}
\newenvironment{Shaded}{\begin{snugshade}}{\end{snugshade}}
\newcommand{\AlertTok}[1]{\textcolor[rgb]{0.75,0.01,0.01}{\textbf{\colorbox[rgb]{0.97,0.90,0.90}{#1}}}}
\newcommand{\AnnotationTok}[1]{\textcolor[rgb]{0.79,0.38,0.79}{#1}}
\newcommand{\AttributeTok}[1]{\textcolor[rgb]{0.00,0.34,0.68}{#1}}
\newcommand{\BaseNTok}[1]{\textcolor[rgb]{0.69,0.50,0.00}{#1}}
\newcommand{\BuiltInTok}[1]{\textcolor[rgb]{0.39,0.29,0.61}{\textbf{#1}}}
\newcommand{\CharTok}[1]{\textcolor[rgb]{0.57,0.30,0.62}{#1}}
\newcommand{\CommentTok}[1]{\textcolor[rgb]{0.54,0.53,0.53}{#1}}
\newcommand{\CommentVarTok}[1]{\textcolor[rgb]{0.00,0.58,1.00}{#1}}
\newcommand{\ConstantTok}[1]{\textcolor[rgb]{0.67,0.33,0.00}{#1}}
\newcommand{\ControlFlowTok}[1]{\textcolor[rgb]{0.12,0.11,0.11}{\textbf{#1}}}
\newcommand{\DataTypeTok}[1]{\textcolor[rgb]{0.00,0.34,0.68}{#1}}
\newcommand{\DecValTok}[1]{\textcolor[rgb]{0.69,0.50,0.00}{#1}}
\newcommand{\DocumentationTok}[1]{\textcolor[rgb]{0.38,0.47,0.50}{#1}}
\newcommand{\ErrorTok}[1]{\textcolor[rgb]{0.75,0.01,0.01}{\underline{#1}}}
\newcommand{\ExtensionTok}[1]{\textcolor[rgb]{0.00,0.58,1.00}{\textbf{#1}}}
\newcommand{\FloatTok}[1]{\textcolor[rgb]{0.69,0.50,0.00}{#1}}
\newcommand{\FunctionTok}[1]{\textcolor[rgb]{0.39,0.29,0.61}{#1}}
\newcommand{\ImportTok}[1]{\textcolor[rgb]{1.00,0.33,0.00}{#1}}
\newcommand{\InformationTok}[1]{\textcolor[rgb]{0.69,0.50,0.00}{#1}}
\newcommand{\KeywordTok}[1]{\textcolor[rgb]{0.12,0.11,0.11}{\textbf{#1}}}
\newcommand{\NormalTok}[1]{\textcolor[rgb]{0.12,0.11,0.11}{#1}}
\newcommand{\OperatorTok}[1]{\textcolor[rgb]{0.12,0.11,0.11}{#1}}
\newcommand{\OtherTok}[1]{\textcolor[rgb]{0.00,0.43,0.16}{#1}}
\newcommand{\PreprocessorTok}[1]{\textcolor[rgb]{0.00,0.43,0.16}{#1}}
\newcommand{\RegionMarkerTok}[1]{\textcolor[rgb]{0.00,0.34,0.68}{\colorbox[rgb]{0.88,0.91,0.97}{#1}}}
\newcommand{\SpecialCharTok}[1]{\textcolor[rgb]{0.24,0.68,0.91}{#1}}
\newcommand{\SpecialStringTok}[1]{\textcolor[rgb]{1.00,0.33,0.00}{#1}}
\newcommand{\StringTok}[1]{\textcolor[rgb]{0.75,0.01,0.01}{#1}}
\newcommand{\VariableTok}[1]{\textcolor[rgb]{0.00,0.34,0.68}{#1}}
\newcommand{\VerbatimStringTok}[1]{\textcolor[rgb]{0.75,0.01,0.01}{#1}}
\newcommand{\WarningTok}[1]{\textcolor[rgb]{0.75,0.01,0.01}{#1}}
\setlength{\emergencystretch}{3em} % prevent overfull lines
\providecommand{\tightlist}{%
  \setlength{\itemsep}{0pt}\setlength{\parskip}{0pt}}
\setcounter{secnumdepth}{5}
\ifLuaTeX
  \usepackage{selnolig}  % disable illegal ligatures
\fi

\title{biblatex-unified}
\author{Kai von Fintel (fintel@mit.edu)}
\date{2022-02-05}

\begin{document}
\maketitle

\hypertarget{introduction-and-usage}{%
\section{Introduction and Usage}\label{introduction-and-usage}}

\texttt{biblatex-unified} is an opinionated biblatex implementation of
the
\href{https://www.linguisticsociety.org/resource/unified-style-sheet}{Unified
Stylesheet for Linguistics Journal}. The stylesheet was developed by
\href{https://www.linguisticsociety.org/about/who-we-are/committees/editors-linguistics-journals-celxj}{CELxJ,
the Committee of Editors of Linguistics Journals}.

The first implementation of the stylesheet for LaTeX was
\href{https://raw.githubusercontent.com/semprag/tex/master/sp.cls}{\texttt{sp.bst}},
developed for the house-style of the journal
\href{http://semprag.org}{\emph{Semantics and Pragmatics}}
(\emph{S\&P}). Bridget Samuels produced a revised version
\texttt{unified.bst}, which has largely disappeared from the internet.

The current project is a ground-up re-implementation of the unified
stylesheet in modern biblatex. It has been used by \emph{S\&P} in
production for several years.

Please
\href{https://github.com/semprag/biblatex-sp-unified/issues/new}{file an
issue at github} to let us know of any problems you encounter and any
recommendations for improvement.

\hypertarget{sources}{%
\subsection{Sources}\label{sources}}

The \texttt{biblatex-unified} style consists of two files:

\begin{itemize}
\tightlist
\item
  \href{unified.bbx}{\VERB|\NormalTok{unified.bbx}|} -- for formatting
  the bibliography.
\item
  \href{unified.cbx}{\VERB|\NormalTok{unified.cbx}|} -- for formatting
  in-text citations in the style of \emph{S\&P}.

  \begin{itemize}
  \tightlist
  \item
    Since the Unified Stylesheet does not give any guidelines for
    in-text citations, this file is optional and users can choose other
    citation styles, such as the \texttt{authoryear-comp} citation style
    that comes with biblatex.
  \end{itemize}
\end{itemize}

\hypertarget{requirements-and-backward-compatibility}{%
\subsection{Requirements and backward
compatibility}\label{requirements-and-backward-compatibility}}

Compiling LaTeX documents with this style depends on a fairly recent TeX
installation that includes biblatex 2.0+. It is tested only with the
\texttt{biber} backend. TexLive 2019 or later would be ideal.

\hypertarget{installation}{%
\subsection{Installation}\label{installation}}

Manual installation involves putting the two files \texttt{unified.bbx}
and \texttt{unified.cbx} somewhere where your TeX system can find them.
This could be local in the same directory as your tex source file, or in
your \texttt{TEXMF} ``home'' directory, or in the system \texttt{TEXMF}
directories.

\hypertarget{configuring-your-.tex-document-to-use-the-style}{%
\subsection{\texorpdfstring{Configuring your \texttt{.tex} document to
use the
style}{Configuring your .tex document to use the style}}\label{configuring-your-.tex-document-to-use-the-style}}

To use the style in conjunction with \emph{S\&P}'s
\href{https://raw.githubusercontent.com/semprag/tex/master/sp.cls}{\VERB|\NormalTok{sp.cls}|},
simply add the \texttt{biblatex} class option when importing
\VERB|\NormalTok{sp.cls}|:

\begin{Shaded}
\begin{Highlighting}[]
    \BuiltInTok{\textbackslash{}documentclass}\NormalTok{[biblatex]\{}\ExtensionTok{sp}\NormalTok{\}}
\end{Highlighting}
\end{Shaded}

If you are not using the \emph{S\&P} document class, you can still use
this style by adding the following to your preamble (after
\VERB|\BuiltInTok{\textbackslash{}documentclass}\NormalTok{\{}\ExtensionTok{...}\NormalTok{\}}|
but before
\VERB|\KeywordTok{\textbackslash{}begin}\NormalTok{\{}\ExtensionTok{document}\NormalTok{\}}|):

\begin{Shaded}
\begin{Highlighting}[]
    \BuiltInTok{\textbackslash{}usepackage}\NormalTok{[backend=biber,}
\NormalTok{                style=unified,}
\NormalTok{                maxcitenames=3,}
\NormalTok{                maxbibnames=99]\{}\ExtensionTok{biblatex}\NormalTok{\}}
\end{Highlighting}
\end{Shaded}

The unified citation style relies on hyperlinking between in-text
citations and the bibliography. So, the \texttt{hyperref} package is
required. It is automatically loaded by \texttt{sp.cls} but if you use a
different document class and \texttt{hyperref} is not loaded by that
class, you need to add \texttt{\textbackslash{}usepackage\{hyperref\}}
to your preamble as well.

If you were previously using \VERB|\NormalTok{natbib}|, remove
\VERB|\BuiltInTok{\textbackslash{}usepackage}\NormalTok{\{}\ExtensionTok{natbib}\NormalTok{\}}|
and any accompanying
\VERB|\BuiltInTok{\textbackslash{}bibliographystyle}\NormalTok{\{}\ExtensionTok{...}\NormalTok{\}}|
and \VERB|\FunctionTok{\textbackslash{}bibpunct}\NormalTok{\{...\}}|
settings. You might also find it helpful to add
\VERB|\NormalTok{natbib}| to the list of options
(\VERB|\BuiltInTok{\textbackslash{}usepackage}\NormalTok{[..., natbib]\{}\ExtensionTok{biblatex}\NormalTok{\}}|),
to load biblatex's \VERB|\NormalTok{natbib}| compatibility module, which
implements common \VERB|\NormalTok{natbib}| commands like
\VERB|\KeywordTok{\textbackslash{}citet}|,
\VERB|\KeywordTok{\textbackslash{}citep}|,
\VERB|\FunctionTok{\textbackslash{}citealt}|,
\VERB|\FunctionTok{\textbackslash{}citealp}|, etc.

Whether you're using \texttt{sp.cls} or a different document class,
you'll need to change the usual BibTeX commands to biblatex, in two
places:

\begin{enumerate}
\def\labelenumi{\arabic{enumi}.}
\tightlist
\item
  Replace the
  \VERB|\BuiltInTok{\textbackslash{}bibliography}\NormalTok{\{}\ExtensionTok{your{-}bibfile}\NormalTok{\}}|
  line in the backmatter with
  \VERB|\FunctionTok{\textbackslash{}printbibliography}|.
\item
  Add the following command to your preamble:\\
  \VERB|\FunctionTok{\textbackslash{}addbibresource}\NormalTok{\{your{-}bibfile.bib\}}|

  \begin{itemize}
  \tightlist
  \item
    NB: the \VERB|\NormalTok{.bib}| extension must be included (unlike
    BibTeX)
  \end{itemize}
\end{enumerate}

\hypertarget{testing}{%
\subsection{Testing}\label{testing}}

Testing consists of rendering \VERB|\NormalTok{unified{-}test.tex}| (and
\VERB|\NormalTok{unified{-}test.bib}|) into a PDF, comparing the result
to the sample bibliography in the Unified Stylesheet guidelines.

\hypertarget{producing-the-documentation}{%
\subsection{Producing the
documentation}\label{producing-the-documentation}}

The documentation consists of the \VERB|\ExtensionTok{README.md}| file
and this PDF. The PDF can be regenerated by typesetting the
\VERB|\ExtensionTok{tex}| source file with
\VERB|\ExtensionTok{xelatex}|. The \VERB|\ExtensionTok{tex}| is actually
itself generated from the underlying \VERB|\ExtensionTok{md}| markdown
file via \VERB|\ExtensionTok{pandoc}| with the following incantation:

\begin{Shaded}
\begin{Highlighting}[]
\ExtensionTok{pandoc} \AttributeTok{{-}f}\NormalTok{ markdown }\AttributeTok{{-}t}\NormalTok{ latex biblatex{-}unified.md }\AttributeTok{{-}s}\DataTypeTok{\textbackslash{} }
\ExtensionTok{{-}o}\NormalTok{ biblatex{-}unified.tex }\AttributeTok{{-}{-}highlight{-}style}\OperatorTok{=}\NormalTok{kate}
\end{Highlighting}
\end{Shaded}

\hypertarget{implementation-notes}{%
\section{Implementation notes}\label{implementation-notes}}

\texttt{biblatex-unified} is meant to implement as fully as possible the
\href{http://celxj.org/downloads/UnifiedStyleSheet.pdf}{Unified Style
Sheet for Linguistics Journals}, which consists of a number of
guidelines and thoughts on the formatting of bibliographies, a sample
bibliography and some comments on that sample.

The \texttt{biblatex-unified} distribution includes a copy of the
Unified Style Sheet and a bib-file that corresponds to the references
typeset in the Style Sheet specifications. So, this bib-file can be used
to test the implementation of the Style Sheet. We include the source
code of the test LaTeX file and the resulting pdf to demonstrate that we
have faithfully implemented the Style Sheet, with some minor
differences.

When it comes to implementing the guidelines in a LaTeX typesetting
environment, most of the guidelines can be implemented in the
bibliography style (a bst file for BibTex or a bbx file for biblatex),
but there are some points that need to be taken into account earlier in
the preparation of the bib file containing the bibliographic details for
the references to be listed.

Here is a point by point copy of the Unified Style Sheet guidelines,
followed by commentary on our implementation.

\begin{quote}
1. \textbf{Superfluous font-styles should be omitted.} Do not use small
caps for author/editor names, since they do not help to distinguish
these from any other bits of information in the citation. In contrast,
italics are worthwhile for distinguishing volume (book, journal,
dissertation) titles {[}+ital{]} from article and chapter titles
{[}-ital{]}.
\end{quote}

\begin{quote}
2. \textbf{Superfluous punctuation should be left out.} Once italic is
adopted to distinguish volumes from articles/chapters (as above), then
single or double quotations around article titles are superfluous and
only add visual clutter.
\end{quote}

\texttt{biblatex-unified}: Article and chapter titles are set in roman
font, while titles of books, dissertations, and journals are set in
italics. There are no quotation marks around titles.

\begin{quote}
3. \textbf{Differing capitalization styles should be used to make
category distinctions.} Use capitalization of all lexical words for
journal titles and capitalize only the first word (plus proper names and
the first word after a colon) for book/dissertation titles and
article/chapter titles. This is a useful diagnostic for discriminating
between titles that are recurring and those that are not. The journal
style for capitalization should also be applied to the title of book
series. Thus, the citation of a SNLLT volume would be punctuated:
\emph{Objects and other subjects: Grammatical functions, functional
categories, and configurationality} (Studies in Natural Language and
Linguistic Theory 52).
\end{quote}

\texttt{biblatex-unified}: Recurring titles (i.e.~titles of journals and
book series) are set in title case (capitalization of all lexical
words), while non-recurring titles (articles, chapters, books,
dissertations) are set in sentence case (capitalize only the first word,
plus proper names and the first word after a colon).

NB: It is fairly easy to convert a title given in title case into
sentence case. But the other way round is quite difficult.
\texttt{biblatex-unified.bbx} depends on the bib file containing title
case for those elements that should be set in title case. For those
elements that will be set in sentence case, the bbx file will do the
conversion to sentence case. Proper names (and the first word after a
colon) need to be protected against that lower casing, so the bib-file
should have \texttt{\{...\}} braces around those words. Please note that
just putting \texttt{\{\}} around the first letter in words that should
be capitalized is bad practice since it prevents proper kerning of the
space between the first and second letter. Even worse practice is to
``double wrap'' a whole title in braces, which prevents any style file
from setting the title in the style specified by its formatting
philosophy.

NB: It is to be encouraged that titles and subtitles be entered into
separate fields in the bib entry. In that case, the bbx file will
properly capitalize the first letter of the subtitle. So, the first of
the following formats is preferred for use with the unified bbx style
(but it has the disadvantage of not necessarily being compatible with
other bibliography styles):

Option 1

\begin{verbatim}
title = {Government and Binding Theory and the Minimalist Program},
subtitle = {Principles and Parameters in Syntactic Theory},
\end{verbatim}

Option 2

\begin{verbatim}
title = {Government and Binding Theory and the Minimalist Program:  
        {Principles} and Parameters in Syntactic Theory},
\end{verbatim}

\begin{quote}
4. \textbf{All author/editor first names should be spelled out.} Not
doing so only serves to make the citation less informative. Without full
first names, the 20th century index for Language alone would conflate
five different people as `J. Smith', four as `J. Harris', three each
under `A. Cohen' and `P. Lee', two each under `R. Kent', `J. Anderson',
`H. Klein' and `J. Klein'.
\end{quote}

\texttt{biblatex-unified}: The style does not abbreviate first names.
Compliance with the Unified Style depends on the bib-file containing
full names for everyone.

\begin{quote}
5. \textbf{The ampersand is useful.} Use ampersand to distinguish higher
and lower order conjuncts, i.e.~{[}W \& X{]} and {[}Y \& Z{]}, as in
Culicover \& Wilkins and Koster \& May. It is relatively easy to see
that reference is made here to two pairs of authors here (cf.~Culicover
and Wilkins and Koster and May).
\end{quote}

\texttt{biblatex-unified}: The style uses the ampersand rather than a
final ``and'' in author and editor lists. This is taken care of by
\texttt{biblatex-unified}.

\begin{quote}
6. \textbf{Name repetitions are good.} While using a line \_\_\_\_ may
save a little space, or a few characters, it also makes each such
citation referentially dependent on an antecedent, and the effort of
calculating such antecedents is more than what it saved typographically.
Each citation should be internally complete.
\end{quote}

\texttt{biblatex-unified}: The style does not use a dashed line for
repeated authors in the reference list.

\begin{quote}
7. \textbf{Four digit year plus period only.} Extra parentheses are
visual clutter and superfluous.
\end{quote}

\texttt{biblatex-unified}: The style complies with the Unified Style
Sheet.

\begin{quote}
8. \textbf{Commas and periods and other punctuation.} Separate citation
components with periods (e.g., Author. Year. Title.) and subcomponents
with commas (e.g., Author1, Author2 \& Author3). Please note the
ampersand (\&), rather than the word ``and'' before the name of the last
author, and no comma before the ``\&''. The use of the colon between
title and subtitle and between place and publisher is traditional, but
we do not use it between journal volumenumber and pagenumbers.
\end{quote}

\texttt{biblatex-unified}: The style complies with the Unified Style
Sheet.

\begin{quote}
9. \textbf{Parentheses around ed.~makes sense.} Commas and periods
should be used exclusively to separate citation components (e.g.,
``Author. Year.''), or subcomponents (e.g.~``author1, author2 \&
author3). Since''ed.'' is neither a component nor a subcomponent, but a
modifier of a component, it should not be separated from the name by a
comma:
\end{quote}

\begin{quote}
surname, firstname = author surname, firstname (ed.). = editor (NOT
surname, firstname, ed.) surname, firstname \& firstname surname (eds.)
= editors
\end{quote}

\texttt{biblatex-unified}: The style complies with the Unified Style
Sheet.

\begin{quote}
10. \textbf{For conference proceedings, working papers, etc.} For
conference proceedings published with an ISSN, treat the proceedings as
a journal: Include both the full conference name and any commonly used
acronym for the conference (BLS, WCCFL, etc.) in the journal title
position. For proceedings not published with an ISSN, treat the
proceedings as any other book, using the full title as listed on the
front cover or title page. If the title (and subtitle if there is one)
only includes an acronym for the conference name, expand the acronym in
square brackets or parentheses following the acronym. If the title does
not include an acronym which is commonly used for the conference name,
include the acronym in square brackets or parentheses following the
conference name. The advantage of including the acronym after the
society title is that it makes the entry much more identifiable in a
list of references.
\end{quote}

{[}From the comments by Joseph Salmons:{]}

\begin{quote}
Do not include ``proceedings of the'' or ``papers from the''.
\end{quote}

NB: The examples included with the Unified Style Sheet do not include
the editors of the CLS proceedings volume. We agree with this practice,
since the names of editors of such proceedings are often hard to get
hold of.

\texttt{biblatex-unified}: These guidelines need to be satisfied in the
construction of the bib-file. There are two approaches, one more of a
hack than the other:

\emph{Option 1}: Use the \texttt{@inproceedings} entry type and include
the ISSN of the proceedings when available. Example:

\begin{verbatim}
@inproceedings{casali:1998a,
    Author = {Casali, Roderic F.},
    Booktitle = {Chicago Linguistic Society (CLS)},
    ISSN = {0577-7240},
    Number = {1},
    Pages = {55-68},
    Title = {Predicting {ATR} Activity},
    Volume = {34},
    Year = {1998}}
\end{verbatim}

Comment: Following the Unified Style Sheet, we can let the presence of
an ISSN control whether an article in conference proceedings is set like
a journal article or like a book chapter. To do this, the bib-file needs
to contain the ISSN of proceedings where possible. When the ISSN is in
the bib-file, \texttt{biblatex-unified} will not print it with the entry
but it will trigger setting as an article. If there's no ISSN in the
entry, \texttt{biblatex-unified} will set it as a book chapter. The ISSN
can be found through \href{http://www.worldcat.org}{WorldCat} fairly
easily. Here's a list of common conference proceedings with their ISSN:

\begin{itemize}
\tightlist
\item
  BLS: 0363-2946
\item
  CLS: 0577-7240
\item
  NELS: 0883-5500
\item
  SALT: 2163-5951
\item
  WCCFL: 1042-1068
\end{itemize}

\emph{Option 2}: Another option, less portable and definitely a hack, is
to use the \texttt{@article} entry type, putting the conference name
(and the acronym in parentheses; no need to case protect the acronym,
since journal titles are set as is) in the \texttt{journal} field, and
do not list the editors. Example:

\begin{verbatim}
@article{casali:1998b,
    Author = {Casali, Roderic F.},
    Journal = {Chicago Linguistic Society (CLS)},
    Number = {1},
    Pages = {55-68},
    Title = {Predicting {ATR} Activity},
    Volume = {34},
    Year = {1998}}
\end{verbatim}

\begin{quote}
11. \textbf{Use ``edn.'' as an abbreviation for ``edition'', thus ``2nd
edn.''.} This avoids ambiguity and confusion with ``ed.'' (editor).
\end{quote}

\texttt{biblatex-unified}: The style complies with the Unified Style
Sheet.

\begin{quote}
12. \textbf{Names with ``von'', ``van'', ``de'', etc.} If the ``van''
(or the ``de'' or other patronymic) is lower case and separated from the
rest by a space (e.g.~Elly van Gelderen), then alphabetize by the first
upper-case element:
\end{quote}

\begin{quote}
\texttt{Gelderen,\ Elly\ van}
\end{quote}

\begin{quote}
The addition of ``see \ldots{}'' in comprehensive indices and lists
might be helpful for clarification:
\end{quote}

\begin{quote}
\texttt{van\ Gelderen,\ Elly\ (see\ Gelderen)}
\end{quote}

\texttt{biblatex-unified}: This is a point where we depart from the
Unified Style Sheet. If the ``van'' (or the ``de'' or other patronymic)
is lower case and separated from the rest by a space (e.g.~Elly van
Gelderen), then alphabetize by the first upper-case element \textbf{but
display the particle as part of the last name in its usual position}:

\begin{verbatim}
`van Gelderen, Elly [alphabetized under "G"]`
\end{verbatim}

NB: biblatex-unified is the only biblatex style that we know of that
implements this practice (it took some hacking to make this possible).
There is no need to do anything in the bib-file to ensure proper
treatment of ``von'' etc.

\begin{quote}
13. \textbf{Names with ``Jr.'', ``IV.'', etc.} Following library
practice, list elements such as ``Jr.'' as a subelement after names,
separated by a comma.
\end{quote}

\texttt{biblatex-unified}: The style complies with the Unified Style
Sheet.

\begin{quote}
14. \textbf{Use ``In'' to designate chapters in collections.} This makes
the book's format maximally similar to the standard citation format.
This, in turn, would be time-saving when the author or the editor notice
that more than one article is cited from a given collection and hence
that that book's details should be set out as a separate entry in the
references (and the full details deleted from the articles' entries).
author. year. chaptertitle. In editorname (ed.), collectiontitle,
pagenumbers. publisher.
\end{quote}

\texttt{biblatex-unified}: The style complies with the Unified Style
Sheet.

\begin{quote}
15. \textbf{Journal volume numbers.} We favor:
volumenumber(volumeissue). startingpage--endingpage. Thus: 22(1).
135-169. Note the space between volume number/issue and page numbers.
Special formatting (e.g., bold for volume number) is superfluous. Issue
numbers are a parenthetical modifier (cf.~``ed.'' above) of the volume
number. While it is not NECESSARY information for identifying the
article, it is extremely USEFUL information.
\end{quote}

\texttt{biblatex-unified}: The style complies with the Unified Style
Sheet. Providing issue numbers is a guideline for the preparation of the
bib-file.

\begin{quote}
16. \textbf{Dissertations/theses.} These conform to the
already-widespread Place: Publisher format and fit readily into the rest
of the standard: Cambridge, MA: MIT dissertation. Instead of archaic
state abbreviations, use the official two-letter postal abbreviations.
Note that national and other traditions vary in exactly what is labeled
`thesis' versus `dissertation' and in distinguishing `PhD' from
`doctoral' dissertations. Cambridge, MA: MIT dissertation. Chapel Hill:
UNC MA thesis.
\end{quote}

\texttt{biblatex-unified}: The style complies with the Unified Style
Sheet. Using the two-letter postal abbreviations is something that needs
to be taken care of during the preparation of the bib-file.

\begin{quote}
17. \textbf{On-line materials.} The basic information here -- author,
date, title -- remains the same, and the URL where the resource was
found takes the place of publisher or journal. We urge authors to
include the date the material was accessed, in parentheses after the
URL, since new versions often replace old ones. For a .pdf file, this
would be the date of downloading, but for a resource like an on-line
dictionary consulted repeatedly, a range of dates may be needed. For
additional discussion of handling online citations, authors may want to
consult this guide:
\end{quote}

\begin{quote}
Walker, Janice R. \& Todd Taylor. 1998. The Columbia Guide to Online
Style. New York: Columbia University Press.
\end{quote}

\texttt{biblatex-unified}: We take it that this does not refer to
material that has officially been published online, in which case
permanent document identifiers such as DOIs and the like will take care
of the linking requirements. Rather, we are assuming this refers to
``unpublished'' material available online (on author's homepages or
repositories such as the Semantics Archive or LingBuzz). In this case,
just use the \texttt{@unpublished} entry-type, give the URL in the
\texttt{url\ =\ \{\}} field, and if deemed necessary include the date
the material was accessed by adding an \texttt{urldate\ =\ \{\}} field.
The date should be given in YYYY-MM-DD format,
e.g.~\texttt{urldate\ =\ \{2013-08-11\}}. The style will then add the
date in parentheses, formatted to the style sheet's rather odd format.

\hypertarget{departures-from-the-unified-style-sheet}{%
\section{Departures from the Unified Style
Sheet}\label{departures-from-the-unified-style-sheet}}

We already mentioned our one major disagreement with the Unified Style
Sheet: the incorrect treatment of the ``von'' part of names.

We depart from the style sheet only in a few other places. These are
actually only departures from the example formatting given and do not
pertain to any explicit guidelines.

\begin{enumerate}
\def\labelenumi{\arabic{enumi}.}
\tightlist
\item
  The acronym ``CLS'' given in parentheses after ``Chicago Linguistic
  Society'' in the Casali 1986 entry is not given in italics in the
  example. Since it is part of the ``journal title'' field,
  \texttt{biblatex-unified} will typeset it in italics.
\item
  The example URLs have a period inside the parentheses around the URL
  access date. We have relocated this period to outside the parentheses.
\item
  In the one case of an online journal article, the Pedersen 2005 entry,
  the example has a comma between the journal + volume and the URL. We
  think that this is analogous to the break between journal +
  volume(issue) and page numbers, and so it should be a period instead
  of a comma. That is what \texttt{biblatex-unified} does.
\item
  There is a period between the URL and the URL access date (which is in
  parentheses). We do not think that this period should be there.
  \texttt{biblatex-unified} has a space.
\end{enumerate}

\hypertarget{additional-remarks-on-dois-and-other-links}{%
\section{Additional remarks on DOIs and other
links}\label{additional-remarks-on-dois-and-other-links}}

The Unified Style Sheet was devised before the widespread use of DOIs to
identify the source of materials that are available electronically. It
is good practice for authors to include DOIs in their bib-file for
anything that has a DOI. Most modern publications prominently display
the DOI on the first page of the work and/or in the metadata. Sometimes
a DOI is harder to find, but \href{http://scholar.google.com/}{Google
Scholar} and
\href{http://crossref.org/SimpleTextQuery/}{crossref.org/SimpleTextQuery}
can help. \texttt{biblatex-unified} displays DOIs as a full link (such
as \texttt{https://doi.org/10.3765/sp.10.1}) in accordance with
\href{https://www.crossref.org/display-guidelines/}{the guidelines
imposed by Crossref}. If you would like a more compact display
(\texttt{DOI:10.3765/sp.10.1}), you can achieve this by adding the
\texttt{compactdois} package option to the
\VERB|\BuiltInTok{\textbackslash{}usepackage}| command:

\begin{Shaded}
\begin{Highlighting}[]
    \BuiltInTok{\textbackslash{}usepackage}\NormalTok{[backend=biber,}
\NormalTok{                style=unified,}
\NormalTok{                maxcitenames=3,}
\NormalTok{                maxbibnames=99,}
\NormalTok{                compactdois]\{}\ExtensionTok{biblatex}\NormalTok{\}}
\end{Highlighting}
\end{Shaded}

If you're using the \texttt{sp.cls} document class with the
\texttt{biblatex} option, you should instead add this to the preamble:
\VERB|\FunctionTok{\textbackslash{}ExecuteBibliographyOptions}\NormalTok{\{compactdois\}}|,
if you want compact DOIs

When \texttt{biblatex-unified} is used with the up-todate version of the
\texttt{sp.cls} documentclass, once a bibentry has a DOI, a link in the
\texttt{url} field will not also be displayed. If you want to achieve
this effect with other document classes, you can add the following to
your preamble after loading \texttt{biblatex-unified}:

\begin{Shaded}
\begin{Highlighting}[]
\FunctionTok{\textbackslash{}DeclareSourcemap}\NormalTok{\{}
        \FunctionTok{\textbackslash{}maps}\NormalTok{[datatype=bibtex]\{}
        \FunctionTok{\textbackslash{}map}\NormalTok{\{}\FunctionTok{\textbackslash{}step}\NormalTok{[fieldsource=doi,final]}
             \FunctionTok{\textbackslash{}step}\NormalTok{[fieldset=url,null]}
             \FunctionTok{\textbackslash{}step}\NormalTok{[fieldset=urldate,null]\}\}\}}
\end{Highlighting}
\end{Shaded}

\hypertarget{guidelines-for-the-preparation-of-bib-files}{%
\section{Guidelines for the preparation of bib
files}\label{guidelines-for-the-preparation-of-bib-files}}

For convenience, we repeat the best practices for preparing your bib
file for use with \texttt{biblatex-unified}.

\begin{enumerate}
\def\labelenumi{\arabic{enumi}.}
\tightlist
\item
  Any words in the titles of articles, books, dissertations that should
  not be lowercased (other than the first word) have to be protected by
  \{\} brackets around that word (not just around the first letter of
  the word and never double wrap the entire title!). This includes the
  first word after a subtitle colon (unless the subtitle field is used
  instead).
\item
  Give full first names for all authors and editors (all persons listed
  in the bib file).
\item
  For conference proceedings, working papers, etc. For conference
  proceedings published with an ISSN, treat the proceedings as a
  journal: Include both the full conference name and any commonly used
  acronym for the conference (BLS, WCCFL, etc.) in the journal title
  position. Do not include the editors in the bib-file. For proceedings
  not published with an ISSN, treat the proceedings as any other book,
  using the full title as listed on the front cover or title page. If
  the title (and subtitle if there is one) only includes an acronym for
  the conference name, expand the acronym in square brackets or
  parentheses following the acronym. If the title does not include an
  acronym which is commonly used for the conference name, include the
  acronym in square brackets or parentheses following the conference
  name. The advantage of including the acronym after the society title
  is that it makes the entry much more identifiable in a list of
  references. Do not include ``proceedings of the'' or ``papers from
  the''.
\item
  For on-line materials, you can give the date the resource was
  accessed. Use the \texttt{urldate} field for this purpose and give the
  date in the YYYY-MM-DD format, e.g.~\texttt{2013-08-11}.
\item
  Use two-letter postal abbreviations for all US cities.
\end{enumerate}

\hypertarget{license-and-copyright}{%
\section{License and copyright}\label{license-and-copyright}}

Copyright ©2022 Kai von Fintel.

This package is author-maintained. Permission is granted to copy,
distribute and/or modify this software under the terms of the LaTeX
Project Public License, version 1.3c.

This software is provided ``as is,'' without warranty of any kind,
either expressed or implied, including, but not limited to, the implied
warranties of merchantability and fitness for a particular purpose.

\hypertarget{changelog-and-release-notes}{%
\section{Changelog and release
notes}\label{changelog-and-release-notes}}

\textbf{v1.00 (2020-05-25)} Initial CTAN release.\\
\textbf{v1.01 (2020-09-11)} Added note that \texttt{hyperref} is
required.\\
\textbf{v1.02 (2022-02-05)} Added \texttt{compactdois} option,
documentation on DOIs.

\end{document}
